% Metódy inžinierskej práce

\documentclass[10pt,slovak,a4paper]{article}

\usepackage[slovak]{babel}
\usepackage[IL2]{fontenc} % lepšia sadzba písmena Ľ než v T1
\usepackage[utf8]{inputenc}
\usepackage{graphicx} % na vkladanie obrázkov
\usepackage{url} % príkaz \url na formátovanie URL
\usepackage{hyperref} % odkazy v texte budú aktívne (pri niektorých triedach dokumentov spôsobuje posun textu)
\usepackage{cite} % na správne citovanie
\usepackage{subcaption} % na vedľajšie obrázky
\usepackage{wrapfig} % na plávajúce obrázky
\usepackage{amsmath} % chybajuci usepackage

\pagestyle{headings}

\title{Vplyv odporúčacích systémov na personalizované nakupovanie v elektronickom
obchode}

\author{
\begin{minipage}[t]{0.85\textwidth}
    \vspace{-2ex}
    Roman Dunko\\[2pt]
	{\small Slovenská technická univerzita v Bratislave}\\
	{\small Fakulta informatiky a informačných technológií}\\
	{\small \texttt{xdunko@stuba.sk}}
\end{minipage}
}

\date{\small 10. október 2024}

\begin{document}

\maketitle

\begin{abstract}
\end{abstract}

\section{Úvod}
Vplyv odporúčacích systémov na personalizované nakupovanie v elektronickom obchode \cite{PLP-Framework}

Odporúčacie systémy sú dnes neodmysliteľnou súčasťou elektronického obchodu. Je ťažké nájsť obchod, ktorý nevyužíva tieto výdobytky doby. Tieto systémy využívajú algoritmy na analýzu správania zákazníkov a vytvárajú personalizované odporúčania produktov, ktoré zvyšujú šance na dokončenie nákupu. Mojím cieľom je skúmať, ako tieto systémy ovplyvňujú nákupné správanie spotrebiteľov, a to z pohľadu rozhodovacieho procesu aj celkového zážitku z nakupovania.



\section{Dôležitá časť} \label{dolezita}
Dôležitý obsah článku.

\section{Ešte dôležitejšia časť} \label{dolezitejsia}
Ešte dôležitejší obsah článku.

\section{Záver} \label{zaver}
Záver článku.

\bibliography{literatura} % odkaz na súbor literatura.bib
\bibliographystyle{plain} % štýl bibliografie

\end{document}
